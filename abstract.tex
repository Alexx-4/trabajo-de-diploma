%% Las secciones del "prefacio" inician con el comando \prefacesection{T'itulo}
%% Este tipo de secciones *no* van numeradas, pero s'i aparecen en el 'indice.
%%
%% Si quieres agregar una secci'on que no vaya n'umerada y que *tampoco*
%% aparesca en el 'indice, usa entonces el comando \chapter*{T'itulo}
%%
%% Recuerda que aqu'i ya puedes escribir acentos como: 'a, 'e, 'i, etc.
%% La letra n con tilde es: 'n.

\prefacesection{Resumen}
En el presente trabajo de diploma se implementar\'an mejoras y nuevas funcionalidades para el servidor Open Latino, relacionadas con los mapas tem\'aticos y con el manejo de su configuraci\'on. Basados en la investigaci\'on sobre otros sistemas y librer\'ias que usan mapas tem\'aticos, se decidi\'o que lo m\'as conveniente fuera la implementaci\'on de un nuevo tipo de tem\'atico, al que llamaremos mapa tem\'atico por clasificaci\'on de tipos o de categor\'ias, que es m\'as intuitivo para su utilizaci\'on por parte de los usuarios. Tambi\'en, como parte de este trabajo, se encuentra la creaci\'on de un nuevo pedido que devuelve informaci\'on acerca de los tem\'aticos creados en el sistema. Por \'ultimo, luego de analizar las diferentes interfaces visuales para configuraci\'on de servidores, se lleg\'o a la conclusi\'on de que es necesario una implementaci\'on propia que responda a las necesidades de este servidor en particular, usando poderosos frameworks para este fin, como Angular.

En los cap\'itulos finales se realizan varias pruebas con el fin de verificar el correcto funcionamiento del servidor y de sus nuevas funcionalidades, brindando las conclusiones a las que se arribaron y se propone el trabajo futuro con Open Latino Server (OLS).\\\\

\section*{Palabras Clave}
Mapa tem\'atico, protocolo WMS, Angular, interfaz de usuario, ASP .Net Core, sistema, servidor, backend, frontend, framework, servidor de mapas, Sistemas de Informaci\'on Geogr\'afica (SIG), SIG Web


\prefacesection{Abstract}
This diploma project is proposed to implement improvements and new functionalities for the Open Latino server, related to the thematic maps and managing server configuration. Based on research on other systems and libraries that use thematic maps, it was decided that the most convenient solution to the problem would be the implementation of a new type of thematic, which we will call thematic map by type classification, which is more intuitive for users. Also, as part of this work, the creation of a new request that returns information about the thematics that are created in the system. Finally, after analyzing the different settings visual interfaces of servers, it was reached the conclusion of an own implementation, that responds to the needs of this server in particular, is necessary, using powerful frameworks, like Angular. 

Several tests are performed in the final chapters to verify the correct operation of the server and its new functionalities, providing the conclusions that we arrived and future work with Open Latino Server(OLS).\\\\

\section*{KeyWords}
Thematic map, WMS protocol, Angular, user interface, ASP .Net Core, system, server, backend, frontend, framework, Internet Map Server (IMS), Geografic Information System (GIS), Web GIS