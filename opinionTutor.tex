%% Las secciones del "prefacio" inician con el comando \prefacesection{T'itulo}
%% Este tipo de secciones *no* van numeradas, pero s'i aparecen en el 'indice.
%%
%% Si quieres agregar una secci'on que no vaya n'umerada y que *tampoco*
%% aparesca en el 'indice, usa entonces el comando \chapter*{T'itulo}
%%
%% Recuerda que aqu'i ya puedes escribir acentos como: 'a, 'e, 'i, etc.
%% La letra n con tilde es: 'n.

\prefacesection{Opini'on del Tutor}

{\setlength{\parindent}{0pt} \textbf{Nombre y Apellidos:} MSc. Joanna Campbell Amos}\\
\textbf{Categor\'ia Docente:} Asistente\\
\textbf{Especialidad:} Ciencia de la Computaci\'on\\
\textbf{Centro de trabajo:} Departamento Programaci\'on, Bases de Datos e Ingenier\'ia de Software. Facultad de Matem\'atica y Computaci\'on. Universidad de La Habana

\section*{Contenido del informe}
El estudiante Javier Alejandro Campos Matanzas ha terminado satisfactoriamente  su trabajo de diploma. Para cumplir con los objetivos propuestos, asumi\'o y estudi\'o Open Latino Server2.0 un servidor de mapas multiplataforma que se viene desarrollando en la Casa del software desde el 2016. En este sentido, la labor realizada por \'el fue realmente meritoria debido a que fue necesario que asumiera y dominara un gran volumen de c\'odigo no desarrollado por \'el y poco documentado. Adem\'as realiz\'o una ardua labor investigativa en temas totalmente nuevos para \'el como son los Sistemas de Informaci\'on Geogr\'afica, mapas tem\'aticos y aplicaciones web. Evidencia  de ello se ilustra en los Cap\'itulos 1, 2 y 3 donde recoge el estado del arte relacionado con los objetivos de este trabajo de diploma, tanto desde el punto de vista te\'orico-conceptual, tecnol\'ogico y de resultados previos que constituyen antecedentes de este ejercicio.

El diplomante, en la investigaci\'on y el desarrollo de este trabajo integr\'o muchos de los conocimientos adquiridos durante la carrera, especialmente disciplinas como: Programaci\'on, Base de Datos, e Ingenier\'ia  de Software. Adem\'as, demostr\'o haber adquirido las bases de la metodolog\'ia de la investigaci\'on cient\'ifica, resultado que puede ser valorado en el presente diploma. Se puede decir de Javier Alejandro que est\'a preparado para la nueva vida profesional porque tiene todas las herramientas, capacidad, conocimientos, ganas de aprender, independencia y determinaci\'on para afrontar los nuevos retos profesionales a que se enfrente de aqu\'i en adelante.

Como tutora estoy muy satisfecha con el trabajo de Javier Alejandro y considero que desde el punto de vista acad\'emico, satisface los requerimientos de una tesis de licenciatura y cumple con creces los objetivos que nos trazamos para la misma al inicio de esta etapa.

Por todo lo anterior propongo al tribunal que se le otorgue la calificaci\'on de Excelente (5).

{\setlength{\parindent}{9.5cm} La Habana, diciembre del 2022}

MSc. Joanna Campbell Amos 

\includegraphics[scale=0.2]{images/firma.jpg}
