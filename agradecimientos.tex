%% Las secciones del "prefacio" inician con el comando \prefacesection{T'itulo}
%% Este tipo de secciones *no* van numeradas, pero s'i aparecen en el 'indice.
%%
%% Si quieres agregar una secci'on que no vaya n'umerada y que *tampoco*
%% aparesca en el 'indice, usa entonces el comando \chapter*{T'itulo}
%%
%% Recuerda que aqu'i ya puedes escribir acentos como: 'a, 'e, 'i, etc.
%% La letra n con tilde es: 'n.

\prefacesection{Agradecimientos}

Primero que todo quiero agradecer a mis padres, por haberme forjado como la persona que soy en la actualidad y a los cuales les debo muchos de mis logros, incluido este. Han pasado muchos a\~nos desde que nac\'i y desde entonces, e incluso antes, siempre buscan la forma de ofrecerme lo mejor d\'andome, en ocasiones, lo que no tienen. Por eso, estas l\'ineas no son suficientes para expresar el orgullo que siento por tenerlos como padres. Gracias mam\'a y pap\'a, no hay nadie en este mundo a quien le deba m\'as amor y agradecimiento.

A mis abuelos, mima y pipo, esas personas con cabellos de plata y coraz\'on de oro que siempre estan ah\'i para m\'i, que me animan en los momentos dif\'iciles y me brindan todo su amor cuando m\'as lo necesito. Mima fue mi primera profesora, la que me ense\~n\'o a leer y a escribir y la que me motiv\'o a superarme cada d\'ia m\'as como profesional. Pipo fue mi maestro de vida, con sus consejos y experiencias he sabido afrontar las peores circunstancias. Estoy agradecido con la vida por tenerlos a ambos como familia.

No puede faltar en estas l\'ineas mi futura esposa y compa\~nera de vida, Airelys. Gracias a ella (y Nala, mi mascota) estos \'ultimos a\~nos de mi vida han sido los mejores. Su apoyo, amor y comprensi\'on son uno de los motivos por los cuales he alcanzado mis metas y he afrontado los obst\'aculos. La mejor desici\'on que he tomado ha sido compartir mi vida con ella, gracias por tanto amor y cari\~no.

A mi hermano, que ha estado siempre a mi lado los \'ultimos 15 a\~nos en las buenas y en las malas. Tambi\'en mi abuela Berena, mi t\'ia Licet y mis primos por apoyarme en todo momento y ayudarme en las ocasiones que lo necesit\'e. Gracias por compartir conmigo cada etapa de mi vida. 

A mi t\'io Juanito, sin el cual no estar\'ia donde estoy hoy. Siempre voy a estar agradecido por su ayuda y apoyo incondicional. Igualmente agradecer al resto de mi familia por, de una forma u otra, contruibuir en mi formaci\'on y educaci\'on.

Quiero agradecer a la familia de mi novia, especialmente a Maylin, Oscar y Alejito, por estar siempre al pendiente de m\'i y tratarme como si fuera un hijo m\'as. Tambi\'en me gustar\'ia dar gracias a Abiel, Yami, Elian y Eliany que han aportado su granito de arena para que este sue\~no se hiciera realidad.

A mi tutora por ayudarme en todo desde que entr\'e en la Casa del Software, guiarme y apoyarme durante el proceso de desarrollo de la tesis y todos los consejos brindados. Tambi\'en agradecer al resto de profesores de la facultad por todas las ense\~nanzas, especialmente Alfredo Somoza, por sus \'utiles consejos que estoy seguro me abrir\'an las puertas en el camino en mi vida como profesional y como persona. 

A mis amistades, en especial a Darian, que se ha convertido en un hermano para m\'i, quien me ha ayudado a transitar el camino universitario, celebrando juntos los momentos de felicidad, y d\'andome su hombro en las ocasiones m\'as amargas. Igualmente agradecer a la Galle y Daniela, quienes siempre han estado al pendiente de m\'i. 

A Taymara que ha formado parte de mi vida desde que tengo uso de raz\'on. Mencionar, adem\'as, a B\'arbara, Ros\'angela, Lisandra, Daniel, Yan Carlos, Henry, Pirlo, Nelson y a todas aquellas personas que han intervenido positivamente en mi vida y han contribuido de una forma u otra en la materializaci\'on de este proyecto.\\

A todos muchas gracias.






