%% Los cap'itulos inician con \chapter{T'itulo}, estos aparecen numerados y
%% se incluyen en el 'indice general.
%%
%% Recuerda que aqu'i ya puedes escribir acentos como: 'a, 'e, 'i, etc.
%% La letra n con tilde es: 'n.
\pagenumbering{arabic}
\setcounter{page}{1}

\chapter{Introducci\'on}

Hasta hace algunas d\'ecadas, los mapas se realizaban obteniendo las coordenadas de los puntos por m\'etodos geod\'esicos, topogr\'aficos y astron\'omicos para que despu\'es mediante m\'etodos cartogr\'aficos se elaborara el mapa correspondiente de acuerdo con la proyecci\'on y la escala seleccionada; sin embargo, a partir de la d\'ecada de los 70, con el lanzamiento de los sat\'elites Landsat y con el establecimientos de los procesos cartogr\'aficos digitales, la forma de hacer mapas paulatinamente se volvi\'o m\'as din\'amica. En la actualidad, usando los diferentes sat\'elites, GPS \footnote{Sistema de navegaci\'on y localizaci\'on mediante sat\'elites.} y los Sistemas de Informaci\'on Geogr\'afica (SIG) \footnote{Sistema de informaci\'on capaz de integrar, almacenar, editar, analizar, compartir y mostrar la informaci\'on geogr\'aficamente referenciada.}, el ser humano elabora diferentes mapas que permiten analizar con mayor profundidad la informaci\'on contenida en un territorio.

Un mapa es la representaci\'on gr\'afica de un territorio sobre una superficie bidimensional. Se define tambi\'en como un dibujo o trazado esquem\'atico que representa las caracter\'isticas de un territorio determinado, tales como sus dimensiones, coordenadas, accidentes geogr\'aficos u otros aspectos relevantes. La funci\'on principal de los mapas es brindar
informaci\'on sintetizada sobre puntos de localizaci\'on y coordenadas de orientaci\'on, as\'i como tambi\'en sobre rutas disponibles, caracter\'isticas de la superficie terrestre (relieves, redes fluviales, recursos, etc.), clima regional, l\'imites pol\'itico-territoriales, puntos de inter\'es, distribuci\'on de la poblaci\'on, etc.\cite{mapa}

El siglo XX fue testigo de un avance extraordinario en el desarrollo de la cartograf\'ia, decisivo en este sentido fue el desarrollo de la fotograf\'ia a\'erea y el despliegue satelital hasta llegar al uso de Sistemas de Informaci\'on Geogr\'afica \cite{SIG} (SIG o GIS, por sus siglas en ingl\'es, Geografic Information System).

En las \'ultimas cinco d\'ecadas, los Sistemas de Informaci\'on Geogr\'afica han evolucionado desde un concepto a una ciencia. Esta magn\'ifica evoluci\'on hace que los SIG pasen de ser una herramienta rudimentaria a convertirse en una poderosa
plataforma para comprender y planificar nuestro mundo. El campo de los Sistemas de Informaci\'on Geogr\'afica est\'a marcado por diversos hitos, comenz\'o en los a\~nos sesenta, mientras emerg\'ian las computadoras y los primeros conceptos de geograf\'ia cuantitativa y computacional. El trabajo pionero de Roger Tomlinson \cite{robert} para iniciar, planificar y desarrollar el Sistema de Informaci\'on Geogr\'afica de Canad\'a, dio como resultado el primer SIG computarizado del mundo, en 1963.

El futuro de SIG, junto con su pasaje a la web y computaci\'on en la nube y la integraci\'on con la informaci\'on en tiempo real a trav\'es del internet de las cosas(IoT) \footnote{es el proceso que permite conectar elementos f\'isicos cotidianos al Internet \cite{iot}}, se convirti\'o en una plataforma relevante a casi toda actividad humana, el sistema nervioso del planeta. Mientras que el mundo enfrenta desaf\'ios, tales como la expansi\'on de la poblaci\'on, desforestaci\'on y contaminaci\'on, los SIG \cite{FSIG} jugar\'an un papel cada vez m\'as importante en c\'omo entendemos y abordamos estos problemas y proporcionan el medio para comunicar soluciones utilizando el lenguaje com\'un de los mapas.

El acelerado desarrollo de internet hizo m\'as f\'acil compartir y actualizar la informaci\'on, lo que motiv\'o el surgimiento de los Web GIS, un patr\'on para la implementaci\'on de un GIS moderno, impulsado por un web-servicio estandar que entrega
datos, capacidades y conecta componentes. Un Web GIS puede ser implementado en la nube con permisos o, m\'as com\'un, como una combinaci\'on h\'ibrida obteniendo lo mejor de ambos mundos. Dentro de sus componentes m\'as importantes se encuentra el Servidor de Mapas(en ingl\'es conocido como IMS: Internet Map Server) \cite{webGis}, de gran utilidad para manejar grandes vol\'umenes de informaci\'on provenientes de diversos proveedores y realizar diversas consultas u operaciones (Queries en ingl\'es) sobre los datos que maneja en solo unos milisegundos, incluso si se trata de millones de datos.

A ra\'iz de estos avances tecnol\'ogicos, surgieron nuevas herramientas que facilitan el estudio cient\'ifico a partir de variables diversas y complejas; como ejemplo, se pueden mencionar los mapas de edificios en una ciudad que constituyan hospitales. Su diferencia radica principalmente en la escala del mapa; es decir, generales o detallados, dependiendo del nivel de especializaci\'on al que desee alcanzar el investigador. Esto no es m\'as que un mapa tem\'atico.

Una de las funcionalidades que tambi\'en tienen los IMS es la de dar, precisamente, mapas tem\'aticos, cuyo objetivo es reflejar un aspecto particular de la zona geogr\'afica sobre la que se definen. Pueden centrarse en variables f\'isicas, sociales, pol\'iticas, culturales, econ\'omicas, sociol\'ogicas y cualquier otra relacionada con un territorio concreto. Los mapas tem\'aticos recogen y aportan informaci\'on sobre temas geogr\'aficos peculiares. Pueden ser anal\'iticos si representan un \'unico elemento gr\'afico, o sint\'eticos si re\'unen datos de diferentes mapas. \cite{tematico}

Una diferenciaci\'on necesaria los clasifica en cualitativos (aquellos que representan fen\'omenos sin tener ninguna precisi\'on num\'erica) y cuantitativos (los que representan el valor num\'erico de un fen\'omeno). As\'i pues, nos encontramos ante una gran
variedad de tipos de mapas tem\'aticos. Entre ellos, destacan los de isol\'ineas, que usan l\'ineas curvas para unir puntos de igual valor de un fen\'omeno, los mapas de flujos, que consisten en l\'ineas de diferente espesor para representaciones din\'amicas y los mapas anam\'orficos, que dependen de la magnitud del fen\'omeno representado \cite{tematico}. Es decir, cambian el tama\~no real de los pa\'ises para hacerlo proporcional al hecho que cartograf\'ian. Los mapas tem\'aticos utilizan los mapas topogr\'aficos \footnote{una imagen, generalmente del relieve de la superficie terrestre, a una escala definida.} como mapas base para la representaci\'on gr\'afica de datos de diversa \'indole, lo que se conoce como cartograf\'ia tem\'atica.

Los servidores de mapas son los encargados de servir los Datos Espaciales \footnote{dato que tiene asociada una referencia geogr\'afica de tal modo que se puede localizar exactamente d\'onde sucede dentro de un mapa.} de los territorios, los cuales son agrupados en conjuntos, denominados Capas, que al combinarse dan como resultado los mapas, adem\'as si se le agregan a estas capas un grupo de restricciones se obtiene un mapa tem\'atico \cite{MP}. Estas capas pueden contener
dos tipos de datos: Raster que son cualquier tipo de imagen digital representada en mallas o Vectorial que est\'an destinados a almacenar con gran exactitud los elementos geogr\'aficos. Cada uno de estos est\'a vinculado a una fila en una base de datos que
describe sus atributos y pol\'igonos.

El desarrollo exponencial, a pasos acelerados, de la humanidad y su relaci\'on directa con el perfeccionamiento de los SIG constituyen la raz\'on de ser de esta investigaci\'on, con \'enfasis en los mapas tem\'aticos, haci\'endolos m\'as \'utiles y f\'aciles de manejar por los usuarios, llev\'andolos a un nivel superior, donde se haga m\'as simple su manejo pero a la vez m\'as complejo y espec\'ifico su contenido, mejorando su eficacia.



\section{Formulaci\'on del problema, motivaci\'on y justificaci\'on}

La Casa del Software de la Facultad de Matem\'atica y Computaci\'on de la Universidad de La Habana ha venido desarrollando un Sistema de Informaci\'on Geogr\'afica llamado Open Latino GIS. Entre los componentes de este sistema se encuentra Open Latino Server (OLS). Este servidor fue desarrollado usando las buenas pr\'acticas de la programaci\'on orientada a objetos \cite{OO} y los principios SOLID \cite{solid} por lo que su c\'odigo es f\'acil de entender y permite incorporar mejoras f\'acilmente. Fue creado usando la tecnolog\'ia de ASP.NET Framework \cite{aspnet}, en consecuencia, solo pod\'ia ejecutarse en entorno Windows y publicarse usando IIS \cite{iis}.

Teniendo como objetivo que fuera multiplataforma \footnote{proyectos que operan en m\'ultiples plataformas inform\'aticas.}, se desarroll\'o una versi\'on 2.0 utilizando NET CORE \cite{netcore}. La nueva versi\'on de OLS complet\'o la implementaci\'on del protocolo WMS \footnote{El protocolo est\'andar WMS permite, a grandes rasgos, servir im\'agenes georeferenciadas a trav\'es de internet} (Web Map Service) \cite{wms} que se encontraba pendiente en su versi\'on anterior. En este se encuentran una serie de pedidos b\'asicos entre los que se puede destacar GetMap, GetCapabilities,etc. Este servidor cuenta, adem\'as, con un pedido para visualizar mapas tem\'aticos.

A pesar de las muchas funcionalidades con las que se cuenta, se presenta el problema de que este servidor carece de una interfaz de usuario\footnote{Punto de interacci\'on y comunicaci\'on humano-computadora en un dispositivo.} amigable, para poder configurarlo es necesario hacerlo a nivel de c\'odigo, mediante el uso de programas para modificar las bases de datos. OLS tampoco cuenta con una funcionalidad que permita conocer los detalles acerca de los tem\'aticos que est\'an definidos en este, actualmente se deben realizar consultas sql \footnote{Structured Query Language, lenguaje de programaci\'on que te permite manipular y descargar datos de una base de datos} para conocer esa informaci\'on. Por \'ultimo, los tem\'aticos que existen son por consulta, es decir, la definici\'on de nuevos tem\'aticos se realiza usando consultas de sql.

Resulta de importancia agregar las funcionalidades anteriores al servidor ya que al usuario final se le dificulta el trabajo con el sistema, debido a que debe tener nociones elementales de programaci\'on y conociminetos b\'asicos de sql para manipular la configuraci\'on de OLS e interactuar con las funcionalidades referentes a los mapas tem\'aticos. Se hace necesario, entonces, la implementaci\'on de una interfaz de usuario amigable para el trabajo con la configuraci\'on del servidor y los mapas tem\'aticos, as\'i como el desarrollo de un nuevo tipo de mapas tem\'aticos, m\'as intuitivo, de clasificaci\'on por tipos o de categor\'ias. Adem\'as, de la creaci\'on de un nuevo pedido que devuelva toda la informaci\'on relacionada a los tem\'aticos definidos en OLS.

Para dar soluci\'on a las cuestiones planteadas anteriormente, es necesario dar respuesta a las siguientes preguntas:

\begin{itemize}
\item ?`Existen en la actualidad soluciones que permitan representar la informaci\'on estad\'istica con la que cuentan las personas en un \'area geoespacial determinada?
\item ?`Existen soluciones para la representaci\'on de la informaci\'on estad\'istica que poseen los usuarios sobre el espacio geogr\'afico de su inter\'es que sea capaz de integrar disponibilidad, accesibilidad y robustez, para que los usuarios cuenten
con un mecanismo eficiente que les apoye en la formaci\'on de elementos de juicio para la toma de decisiones?
\item ?`Es posible crear un software que sea de c\'odigo abierto y que adem\'as est\'e disponible estando en l\'inea, que tenga una herramienta para brindar visualizaci\'on sobre el problema en cuesti\'on?
\item ?`Qu\'e soluciones y tecnolog\'ias existentes son apropiadas para la implementaci\'on de la presente investigaci\'on?
\end{itemize}

Es posible dar soluci\'on a las preguntas anteriores gracias a los conocimientos, herramientas y habilidades de investigaci\'on adquiridas en el transcurso de los estudios universitarios en la carrera de Ciencias de la Computaci\'on de la Universidad de La Habana. Se cuenta, adem\'as, con las tecnolog\'ias necesarias, ya se han puesto en uso en la plataforma OLS que se presta como antecedente a los nuevos cambios propuestos, adem\'as se cuenta con aplicaciones precedentes en el propio Departamento de Programaci\'on, Ingenier\'ia de Software y Base de Datos como son C-sharp, ASP.Net Core, SQL Server entre otras que evidencian una rica experiencia de base.


\section{Objetivos}
Para resolver el problema anunciado anteriormente se plantean los siguientes objetivos principales:

\begin{enumerate}
\item Desarrollar una herramienta sobre la plataforma de c\'odigo abierto OLS que presente un mecanismo para visualizar y configurar un nuevo tipo de mapas tem\'aticos clasificados por valores de forma extensible.
\item Implementar una herramienta que permita consultar informaci\'on acerca de los tem\'aticos existentes en el servidor.
\item Desarrollar una interfaz visual intuitiva para manipular la configuraci\'on del servidor y los mapas tem\'aticos.
\end{enumerate}

Para cumplir con los objetivos principales explicados previamente, se plantean los siguientes objetivos secundarios:

\begin{enumerate}
\item Estudiar precedentes de otras aplicaciones que tengan implementado una herramienta para visualizar y configurar mapas tem\'aticos.
\item Investigar las ventajas, desventajas y analizar la factibilidad de las mejores herramientas para implementar la interfaz visual.
\item Extender la arquitectura de clases con la que ya cuenta OLS que cumpla con los principios del SOLID, que sea extensible y mantenible para continuar con las buenas pr\'acticas de la plataforma OLS para que cualquier implementaci\'on posterior en este tema se pueda utilizar sin necesidad de modificar, aumentando la eficacia y sencillez de futuras mejoras al proyecto.
\end{enumerate}


\section{Estructura del trabajo}
Este trabajo de diploma est\'a compuesto por 5 cap\'itulos y se estructura de la siguiente manera:

\begin{itemize}
\item \textbf {Cap\'itulo 1 :} \qquad El primer cap\'itulo presenta una breve introducci\'on al tema, la motivaci\'on, formulaci\'on del problema y justificaci\'on del mismo, los objetivos principales y secundarios, concluyendo este cap\'itulo con la estructura del trabajo de diploma.
\item \textbf {Cap\'itulo 2 :} \qquad Este cap\'itulo est\'a dedicado al estado del arte, donde se analizar\'an del uso de los diferentes IMS que implementan los mapas tem\'aticos por categor\'ias, as\'i como de las diferentes interfaces visuales que existen para configurar un servidor. Se expondr\'an las ventajas y desventajas de su uso.
\item \textbf {Cap\'itulo 3 :} \qquad En este cap\'itulo se expone el marco te\'orico-conceptual del presente trabajo, donde se explican las soluciones que se dieron a los objetivos, tanto conceptualmente como algunos de sus detalles de implementaci\'on. Adem\'as, se aborda acerca del mecanismo que se implementa para la recuperaci\'on de tem\'aticos y la integraci\'on del nuevo tipo de mapa, as\'i como se expondr\'a la arquitectura usada en la implementaci\'on de la interfaz visual. Por \'ultimo se da una explicaci\'on de las tecnolog\'ias empleadas y d\'onde se usaron.
\item \textbf {Cap\'itulo 4 :} \qquad El cap\'itulo 4 est\'a dedicado a exponer un conjunto de pruebas para mostrar el buen funcionamiento de la nueva versi\'on de OLS y el cumplimineto de los objetivos planteados. En estas pruebas se van planteando los objetivos que se quieren verificar con estas y luego se presentan los resultados alcanzados con cada una.
\item \textbf {Cap\'itulo 5 :} \qquad En el quinto y \'ultimo cap\'itulo se proponen las conclusiones y recomendaciones, temas para trabajo futuro, es decir, temas que quedaron pendientes o se pueden realizar de una mejor manera. En este cap\'itulo aparece, adem\'as, la bibliograf\'ia utilizada.
\end{itemize}


